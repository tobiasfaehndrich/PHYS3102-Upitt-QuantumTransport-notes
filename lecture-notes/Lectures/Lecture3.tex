\ytset{https://www.youtube.com/watch?v=8pZCShDa7jE&list=PLtTPtV8SRcxjedflXwNPSI_fxvxwUCjsd}

\section[Materials for Quantum Transport]{\hyperlink{toc}{Materials for Quantum Transport}}

\subsection{Two-Dimensional Electron Gas (2DEG)}

\begin{itemize}
    \item 2D Electron Gas (2DEG) (\ylink{9:01}):
          \begin{itemize}
              \item Came from industry that was needing faster/smaller devices (note they have large mean free paths i.e. small number of impurities, and high electron mobility).
              \item Often in semiconductor heterostructures, formed at interface.
              \item Also in Metal-oxide-semiconductor (MOS) structures (\ylink{14:38}).
              \item Materials are very well matched at interface so there are few defects.
              \item To fill 2DEG you can either apply gate voltage or use doping.
              \item Electron mobility goes up as temperature goes down (less phonon scattering).
          \end{itemize}

          \begin{figure}[H]
              \centering
              \includegraphics[width=0.5\textwidth]{2deg-band.jpg}
              \caption{2DEG in a semiconductor heterostructure.}
              \label{fig:2DEG}
          \end{figure}

    \item Place metal on surface and then heat it up (anneal) so that it diffuses through the semiconductor and creates a contact to the 2DEG.
    \item To make quantum dots we have metals on surface that we apply (very negative) voltage to (gates) to locally deplete the 2DEG.
          \begin{figure}[H]
              \centering
              \includegraphics[width=0.5\textwidth]{quantum-transport-device-example.png}
              \caption{Example quantum transport device using a 2DEG.}
              \label{fig:quantum-transport-device}
          \end{figure}
\end{itemize}

\subsection{One-Dimensional Systems (Nanowires)}

\begin{itemize}
    \item Metallic or semiconductor nanowires (1D systems) (\ylink{22:50}):
          \begin{itemize}
              \item Bottom-up growth (e.g., VLS process) (\ylink{26:30})
              \item Top-down etching from 2D structures
          \end{itemize}
          \begin{figure}[H]
              \centering
              \includegraphics[width=0.5\textwidth]{vls-nanowire.png}
              \caption{VLS grown nanowire process.}
              \label{fig:nanowire}
          \end{figure}
\end{itemize}

\subsection{Zero-Dimensional Systems (Quantum Dots)}

\begin{itemize}
    \item Semiconducting quantum dots (0D systems) (\ylink{26:50}):
          \begin{itemize}
              \item electron motion strongly confined in all 3 dimensions.
              \item quantization so that these are like artificial atoms.
              \item can be made in 2DEG by applying voltages to gates (top-down).
              \item dominant transport is single electron tunneling.
              \item can also be by pillars etched out of 2deg and also self-assembled (strained induced) InAs dots (bottom-up) (\ylink{29:20}).
          \end{itemize}

    \item Metallic clusters \& semiconductor dots (0D): made by chemists, synthesized in solution and can make different sizes that can emit different colors.
    \item Single molecules and chains of atoms: two electrodes placed with tiny gap and then deposit molecules and some will bridge the gap.
\end{itemize}

\subsection{Carbon-Based Materials}

\begin{itemize}
    \item Carbon base material-systems: fullerenes (1985, 0D "buckyballs"), carbon nanotubes (1991, 1D) (\ylink{50:19}), graphene (2004, 2D) (\ylink{48:09}).
\end{itemize}

\subsection{Density of States}

\begin{itemize}
    \item Free electrons in three dimensions (\ylink{40:29}):

          \begin{itemize}
              \item Schrodinger equation: $-\frac{\hbar^2}{2m}\nabla^2\psi = E\psi$.
              \item The eigenstates are: $\psi_{\vec{k}}(\vec{r}) = \frac{1}{(2 \pi)^3} e^{i \vec{k} \cdot \vec{r}}$ with $\vec{k} = (k_x, k_y, k_z)$.
              \item The energy eigenvalues are: $E_{\vec{k}} = \frac{\hbar^2 \vec{k}^2}{2m} = \frac{\hbar^2}{2m}(k_x^2, k_y^2, k_z^2)$.
              \item The unitary volume of the state in k-space is $(2\pi)^3$.
              \item \item The number of states in a 3D k-space volume $d \vec{k} = dk_x dk_y dk_z$ is given by:

                    \[ g(\vec{k}) d\vec{k} = \frac{2}{(2\pi)^3} d\vec{k} \]

                    where the factor of 2 is for spin degeneracy.

              \item $d\vec{k}$ can be written as:
                    \[ d\vec{k} = \frac{4}{3} \pi ((k+dk)^3 - k^3) = 4 \pi k^2 dk \]

                    leading to:

                    \[ g(\vec{k}) d\vec{k} = \frac{1}{\pi^2} k^2 dk \]

              \item By calculating $k^2$ and $dk/dE$:
                    \[ k^2 = \frac{2mE}{\hbar^2} \quad , \quad \frac{dk}{dE} = \sqrt{\frac{2m}{E \hbar^2}} \frac{1}{2}\]

                    \[ \frac{1}{\pi^2} k^2 dk = \frac{1}{\pi^2} \left( \frac{2mE}{\hbar^2} \right) \frac{dk}{dE} dE = \frac{1}{2\pi^2} \left( \frac{2m}{\hbar^2} \right)^{3/2} \sqrt{E} dE \]

              \item Which is the density of states in 3D for free electrons:

                    \[ \boxed{g(E) = \frac{1}{2\pi^2} \left( \frac{2m}{\hbar^2} \right)^{3/2} \sqrt{E}} \]

          \end{itemize}

    \item Can also do this for 2D and 1D and 0D systems:

          \begin{figure}[H]
              \centering
              \includegraphics[width=0.5\textwidth]{dos.png}
              \caption{Density of states in different dimensions.}
              \label{fig:dos-dimensions}
          \end{figure}
\end{itemize}

\subsection{Band Structure}

\begin{itemize}
    \item Band structure and symmetry points (\ylink{54:18})
    \item Direct vs indirect band gaps
          \begin{itemize}
              \item Wider bands are heavier effective mass
              \item Narrower bands are lighter effective mass
          \end{itemize}


    \item Bands vs sub-bands: sub bands based on excited states in quantum wells (not in 3D bulk material)
\end{itemize}

\subsection{Novel Quantum Materials}

\begin{itemize}
    \item Newish quantum transport materials (\ylink{1:08:29}):
          \begin{itemize}
              \item Bi2Se3 topological insulator
              \item MoS2 (transition metal dichalcogenides)
              \item LAO/STO (lanthanum aluminate/strontium titanate)
          \end{itemize}
\end{itemize}