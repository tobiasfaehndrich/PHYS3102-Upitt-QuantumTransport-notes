\section[Technology]{\hyperlink{toc}{Technology}}

\begin{itemize}
    \item Crystal growth (normally done out of house, so that we ensure it is high quality) $\rightarrow$ nanofabrication (normally done in house because no one wants to do it for you)$\rightarrow$ Low-T electrical measurements (science!)
    \item Molecular Beam Epitaxy (MBE): 
    \begin{itemize}
        \item MBE is performed in an ultra-high vacuum (UHV) chamber.
        \item A heat source evaporates pure elements, which then condense on a heated substrate.
        \item Atoms self-assemble into crystalline layers.
        \item The process involves ballistic (of atoms) deposition (molecular flow).
        \item Temperature is a critical parameter.
        \item Effusion cells are used to carry very pure elements (like Gallium, Arsenic, Aluminum, Indium).
        \item A RHEED (reflection high energy electron diffraction) gun reflects electrons off the surface and reads the pattern with a sensor to monitor growth in real time. Signal oscillates, intensity determined by surface roughness.
    \end{itemize}

    \item Metal Organic Vapor Phase Epitaxy (MOVPE) (or MOCVD):
    \begin{itemize}
        \item Temperature usually 600-1200C
        \item Uses metal-organic precursors (like trimethylgallium, triethylgallium, arsine, phosphine) that decompose on the hot substrate to deposit the desired elements (i.e. organics leave and leave semiconductor elements behind).
        \item Used to make nanowires
    \end{itemize}

    \item Pulsed Laser Deposition (PLD):
    \begin{itemize}
        \item UHV chamber with window
        \item laser shoots at target material, ablates material which deposits on substrate
        \item used to make complex oxides (LAO), HTSCs (YBCO, BSCO etc.), and also semiconductors
    \end{itemize}
    \item nanofabrication: pattern definition (photolithography, Electron beam lithography, and nano imprint lithography), Thin film deposition (evaporation, sputtering using plasma), Etching (chemical etching -- wet and dry, Ion milling)
    \begin{figure}[h]
        \includegraphics[width = 0.5 \linewidth]{photo-lithography.png}
        \centering
        \caption{photolithography process}
    \end{figure}

    \item Excimer Laser Stepper: resolution is around 65nm, can make smaller features possible, fast, very expensive.
    \item Stamping nanostructures: PDSM stamp, nanoimprint, step and flash imprint lithography (around 22nm resolution accross large area, fast, need master, overlay is difficult).
    \item Electron Beam Lithography (EBL): a pattern generator to SEM (beam is scanned across surface), very high resolution (sub 10nm), slow, serial process, need conductive substrate or coating. Several million dollars.
    \item Focused Ion Beam (FIB): use atoms like Ga, to mill away material. Pretty similar properties to EBL.

    \item Nanofabrication journey:
    \begin{itemize}
        \item you dont get unlimited good semiconductor wafers
        \item find defects and figure out if you can avoid them in your device design.
        \item practice with just GaAs at first without the 2DEG
    \end{itemize}
    \item Low-T technology:
    \begin{itemize}
        \item Liquid helium 4 at 4.2K (bath cryostat).
        \item Pumped helium 4 -- 1.5K (note that superfluid transition is at 2.17K).
        \item Helium 3 -- 0.3K (expensive, rare, hard to handle)
        \item Dilution Refrigerator (mix of helium 3 and helium 4) -- 0.01K
        \item Adiabatic demagnetization refrigerator -- 0.001K, use solids and apply magnetic fields to cool down (entropy increases, temperature goes down), no helium 3 required.
    \end{itemize}
    \item Cooling electrons down: electron temperature is not the same thing as the lattice temperature (phonons) i.e. the temp of the fridge. 
    \begin{itemize}
        \item Let electrons have time to cool down by wrapping wires around stages of fridge. Electron-phonon coupling. Add resistor to cool down electrons.
        \item Need to filter out high frequency noise (RF) from room temperature electronics. Use RC filters, copper powder filters, pi-filters, etc.
    \end{itemize}
\end{itemize}

