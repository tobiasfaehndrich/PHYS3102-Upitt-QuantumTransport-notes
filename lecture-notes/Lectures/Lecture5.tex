\section[Ballistic Transport]{\hyperlink{toc}{Ballistic Transport}}

\begin{itemize}
    \item Diffusive transport: electrons scatter multiple times off impurities, phonons, other electrons, etc. Mean free path $l$ much smaller than device length $L$ ($l \ll L$)
    \item Quasi-ballistic transport: electrons scatter a few times, $l \sim L$
\end{itemize}


\begin{itemize}
    \item The Drude model describes diffusive transport, based on the free electron gas picture. Electrical resistance comes from scattering (information is lost when scattering).
    \item This gives us equation for velocity of electron in an electric field and mobility of electron, and conductivity.
    \[ \vec{v}_d = -\frac{e \vec{E} \tau}{m} \]
    \[ \mu_d = \frac{e \tau}{m} \]
    \[ \sigma = \frac{n e^2 \tau}{m} = n e \mu_d \]
    \item Einstein relation: electrons near $E_F$ diffuse due to density gradient.
    
    \item Ohms law and drude model are in diffusive regime
    \item In the ballistic regime, electrons travel through the device without scattering. The resistance comes from the contacts (interfaces) to the device, not from within the device itself.
    \item With non-local measurement (voltage is not measured along the current path to electrical ground) you can measure electron focussing in a magnetic field (famous non-local measurement is quantum hall effect).
    \item Split gate hetero structure in 2DEG GaAs/AlGaAs, apply negative voltage to deplete 2DEG under gates and create a narrow constriction (quantum point contact, QPC). When you make the constriction narrow enough you get quantized conductance in units of $G_0 = \frac{2e^2}{h}$.
    

    \item Constricts to 1D transport, so electrons can only move forward or backward. The conductance is given by the Landauer-Buttiker formula.

    \item QPC are not perfect, if you increase temp the plateaus get washed away. For finite temp we must consider a finite occupation of the energy levels. 

    \item The width of the derivative of the fermi function is $\sim 3.5 k_B T$. If $3.5 k_B T$ is comparable to the sub-band spacing then the plateaus get washed out.
    \item Transmission resonance can also wash out the plateaus. 
    \item Reflection can be from impurities or Backscattering from abrupt constriction at QPC. Tapered/smooth and short potential constriction is better for transmission closer to $T=1$ and $R=0$.
    \item Can measure in nanowires also but you need very high magnetic fields to see quantized conductance.
    \item Perpendicular magnetic field also makes plateaus more stretched out in gate voltage.
    \item Parallel magnetic field splits spin degeneracy and you get $1e^2/h$ plateaus.
    \item 

\end{itemize}