\ytset{https://www.youtube.com/watch?v=dFmmHzMGsfs&list=PLtTPtV8SRcxjedflXwNPSI_fxvxwUCjsd}

\section[Quantum Dots]{\hyperlink{toc}{Quantum Dots}}

\subsection{Introduction and Basic Concepts}

\begin{itemize}
    \item Quantum dots differ from single-electron transistors (SETs) due to their quantum confinement energy (\ylink{0:30}).
    \item Key features: quantized level spacing from quantum confinement, discrete energy levels.
    \item Requirements: relatively low number of electrons (semiconductors), small island.
    \item Can be built on different scales:
          \begin{itemize}
              \item Single molecule quantum dots (nm scale)
              \item Self-assembled quantum dots
              \item Gate-defined GaAs quantum dots
              \item Carbon nanotubes (micron scale)
              \item Graphene quantum dots
          \end{itemize}
    \item Paper: \href{https://pubs.acs.org/doi/10.1021/nl049230s}{Few-electron quantum dots in nanowires by M.T. Bjork et al, Nano Letters 4, 1621 (2004)}
\end{itemize}

\subsection{Quantum Dots vs Classical SETs}

\begin{itemize}
    \item Classical dots: regular spacing of conductance peaks (\ylink{4:27}).
    \item Quantum dots: irregular peak spacing due to additional orbital energy.
    \item Addition energy includes both charging energy and quantum (orbital) energy (\ylink{6:30}).
    \item Current-voltage characteristics (\ylink{10:48}):
          \begin{itemize}
              \item Classical SET: zero current until threshold voltage, then linear current flow
              \item Quantum dot: step-like IV curve reflecting discrete level structure
          \end{itemize}
\end{itemize}

\subsection{Transport Properties}

\begin{itemize}
    \item Resonant transmission in quantum dots can achieve near-unity conductance when barriers are symmetric (\ylink{11:14}).
    \item Coulomb diamonds and shell filling (\ylink{16:00}):
          \begin{itemize}
              \item Electrons occupy discrete energy levels
              \item Behavior similar to artificial atoms
              \item Characteristic patterns in stability diagrams
          \end{itemize}
\end{itemize}

\subsection{Magnetic Field Effects}

\begin{itemize}
    \item Effect of magnetic fields on quantum dot energy levels (\ylink{30:00}):
          \begin{itemize}
              \item Zeeman splitting
              \item Fock-Darwin spectrum
          \end{itemize}
\end{itemize}

\subsection{Excited States}

\begin{itemize}
    \item Excited states analysis using higher bias measurements (\ylink{35:55}).
    \item Can reveal:
          \begin{itemize}
              \item Orbital states
              \item Spin states
              \item Vibrational states
          \end{itemize}
\end{itemize}

\subsection{Double Quantum Dots}

\begin{itemize}
    \item Electrostatics of double quantum dots.
    \item Honeycomb charge stability diagrams (\ylink{50:39}).
    \item Enable study of:
          \begin{itemize}
              \item Very narrow transitions
              \item Co-tunneling processes (\ylink{1:02:42})
          \end{itemize}
\end{itemize}
